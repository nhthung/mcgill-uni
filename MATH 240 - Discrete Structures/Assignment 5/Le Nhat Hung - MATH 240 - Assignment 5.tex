\documentclass[a4paper, 11pt]{article}
\usepackage{amssymb}
\usepackage{enumitem}
\usepackage{fullpage}

\newlist{myEnumerate}{enumerate}{9}
\setlist[myEnumerate,1]{label=\arabic*.}
\setlist[myEnumerate,2]{label=(\alph*)}
\setlist[myEnumerate,3]{label=\roman*.}


\begin{document}

\noindent
\large\textbf{Assignment 5} \hfill \textbf{LE, Nhat Hung} \\
\normalsize MATH 240 \hfill \today

\begin{myEnumerate}
	\item
	\begin{myEnumerate}
    	\item
        \( \\
        \begin{array}{llll}
        	243 & = 235 & + & 8 \\
            235 & = 29 \cdot 8 & + & 3 \\
            8 & = 2 \cdot 3 & + & 2 \\
            3 & = 2 & + & 1
        \end{array} \\ \)
        
        $gcd(235, 243)=1$
        
        \( \\
        \begin{array}{ll}
            3 = 2 + 1 & \Rightarrow 3 - 2 = 1 \\
            		  & \Rightarrow 235 - 29 \cdot 8 - (8 - 2 \cdot 3) = 1 \\
                      & \Rightarrow 235 - 30 \cdot 8 + 2 \cdot 3 = 1 \\
                      & \Rightarrow 235 - 30(243 - 235) + 2(235 - 29 \cdot 8) = 1 \\
                      & \Rightarrow 33 \cdot 235 - 30 \cdot 243 - 58 \cdot 8 = 1 \\
                      & \Rightarrow 33 \cdot 235 - 30 \cdot 243 - 58(243 - 235) = 1 \\
                      & \Rightarrow 91 \cdot 235 - 88 \cdot 243 = 1 \\
                      & \Rightarrow 235^{-1} = 91 $ $ ($mod $ 243)
        \end{array} \\ \)
        
        \( \\
        \begin{array}{ll}
        	235x \equiv 12 $ $ ($mod $ 243) & \Rightarrow x \equiv 12 \cdot 91 $ $ ($mod $ 243) \\
            								& \Rightarrow x \equiv 1092 \equiv 120 $ $ ($mod $ 243) \\
                                            & \Rightarrow x = 120
        \end{array} \)
        
        \item
        $235 = 5 \cdot 47$ \\
        $245 = 5 \cdot 7^{2}$ \\
        $gcd(234, 245) = 5$ \\
        
        Suppose $235x \equiv 12 $ $ ($mod $ 245)$ has a solution.
        
        \( \\
        \begin{array}{ll}
        	235x \equiv 12 $ $ ($mod $ 245) & \Rightarrow 235x = 245q + 12, $ $q \in \mathbb{Z} \\
            								& \Rightarrow 235x - 245q = 12 \\
                                            & \Rightarrow gcd(235, 245) \mid 12 \\
                                            & \Rightarrow 5 \mid 12
        \end{array} \\ \)
        
        But $5 \nmid 12$. \\
        Thus, $235x \equiv 12 $ $ ($mod $ 245)$ has no solutions. \\
        
        \item
        $235x \equiv 10 $ $ ($mod $ 245) \Leftrightarrow 47x \equiv 2 $ $ ($mod $ 49)$ \\
        
        \(
        \begin{array}{llll}
        	49 & = 47 & + & 2 \\
            47 & = 23 \cdot 2 & + & 1
        \end{array} \\ \)
        
        $gcd(47, 49) = 1$ \\
        
        \(
        \begin{array}{ll}
        	47 = 23 \cdot 2 + 1 & \Rightarrow 47 - 23 \cdot 2 = 1 \\
            					& \Rightarrow 47 - 23(49 - 47) = 1 \\
                                & \Rightarrow 24 \cdot 47 - 23 \cdot 49 = 1 \\
                                & \Rightarrow 47^{-1} = 24 $ $ ($mod $ 49)
        \end{array} \\ \)
        
        \( \\
        \begin{array}{ll}
        	235x \equiv 10 $ $ ($mod $ 245) & \Rightarrow 47x \equiv 2 $ $ ($mod $ 49) \\
            								& \Rightarrow x \equiv 2 \cdot 24 $ $ ($mod $ 49) \\
                                            & \Rightarrow x \equiv 48 $ $ ($mod $ 49) \\
                                            & \Rightarrow x = 48 + 49k, $ $ k \in \mathbb{Z} $ and $ 0 \leq x < 245 \\
                                            & \Rightarrow x = 48, 97, 146, 195 $ or $ 244
        \end{array} \\ \)
        
    \end{myEnumerate}
	
    \item
    Let $d_{0},...,d_{k - 1}$ digits of $n \in \mathbb{Z}$. \\
    
    \centerline{$n = \overline{d_{k - 1}d_{k - 2}...d_{1}d_{0}}$}
    
    \( \\
    \begin{array}{ll}
     	n \equiv \overline{d_{k - 1}d_{k - 2}...d_{1}d_{0}} $ $ ($mod $ 11) & \Rightarrow n \equiv 10^{k - 1}d_{k - 1} + 10^{k - 2}d_{k - 2} + ... + 10d_{1} + d_{0} $ $ ($mod $ 11) \\
																			& \Rightarrow n \equiv (-1)^{k - 1}d_{k - 1} + (-1)^{k - 2}d_{k - 2} + ... + (-1)d_{1} + d_{0} $ $ ($mod $ 11) \\
																			& \Rightarrow n \equiv d_{k - 1} + (-d_{k - 2}) + ... + (-d_{1}) + d_{0} $ $ ($mod $ 11) \\
                                        									& \hspace{5mm} $or, if last digit's label is odd:$ \\
																			& \hspace{5mm} n \equiv (-d_{k - 1}) + d_{k - 2} + ... + (-d_{1}) + d_{0} $ $ ($mod $ 11)
    \end{array} \\ \)
    
    Without loss of generality, let the last digit's label be even. \\
    Want to prove: \\
    
    \centerline{\textbf{(1)} $11 \mid [(d_{0} + d_{2} + ... + d_{k - 1}) - (d_{1} + d_{3} + ... + d_{k - 2})] \Leftrightarrow 11 \mid n$ \textbf{(2)}}
    
    \( \\
    \begin{array}{ll}
     	\textbf{(1)} & \Rightarrow (d_{0} + d_{2} + ... + d_{k - 1}) - (d_{1} + d_{3} + ... + d_{k - 2}) \equiv 0 $ $ ($mod $ 11) \\
        			 & \Rightarrow d_{k - 1} + (-d_{k - 2}) + ... + (-d_{3}) + d_{2} + (-d_{1}) + d_{0} \equiv 0 $ $ ($mod $ 11) \\
                     & \Rightarrow n \equiv 0 $ $ ($mod $ 11) \\
                     & \Rightarrow 11 \mid n
    \end{array} \\ \)
    
    \( \\
    \begin{array}{lll}
     	\textbf{(2)} & \Rightarrow n \equiv 0 $ $ ($mod $ 11) \\
        			 & \Rightarrow d_{k - 1} + (-d_{k - 2}) + ... + (-d_{3}) + d_{2} + (-d_{1}) + d_{0} \equiv 0 $ $ ($mod $ 11) \\
                     & \Rightarrow (d_{0} + d_{2} + ... + d_{k - 1}) - (d_{1} + d_{3} + ... + d_{k - 2}) \equiv 0 $ $ ($mod $ 11) \\
                     & \Rightarrow 11 \mid [(d_{0} + d_{2} + ... + d_{k - 1}) - (d_{1} + d_{3} + ... + d_{k - 2})] $ $ & \square
    \end{array} \\ \)
    
    Thus, $11 \mid [(d_{0} + d_{2} + ... + d_{k - 1}) - (d_{1} + d_{3} + ... + d_{k - 2})] \Leftrightarrow 11 \mid n$.

    \item
    \begin{myEnumerate}
    	\item
        $p = 13, q = 17$.
        
        \item
        $(p - 1)(q - 1) = 192$
        
        \( \\
        \begin{array}{llll}
        	192 & = 113 & + & 79 \\
            113 & = 79 & + & 34 \\
            79 & = 2 \cdot 34 & + & 11 \\
            34 & = 3 \cdot 11 & + & 1 \\
        \end{array} \\ \)
        
        \(
        \begin{array}{ll}
        	34 = 3 \cdot 11 + 1 & \Rightarrow 34 - 3 \cdot 11 = 1 \\
            					& \Rightarrow 113 - 79 - 3(79 - 2 \cdot 34) = 1 \\
                                & \Rightarrow 113 - 4 \cdot 79 + 6 \cdot 34 = 1 \\
								& \Rightarrow 113 - 4 \cdot (192 - 113) + 6 \cdot (113 - 79) = 1 \\
								& \Rightarrow 11 \cdot 113 - 4 \cdot 192 - 6 \cdot 79 = 1 \\
								& \Rightarrow 11 \cdot 113 - 4 \cdot 192 - 6 \cdot (192 - 113) = 1 \\
                                & \Rightarrow 17 \cdot 113 - 10 \cdot 192 = 1
        \end{array} \\ \)
        
        $e^{-1} \equiv 113^{-1} \equiv 17$ (mod $192$) \\
        
        Then, $d = 17$ \\
        
        Let $M$ be $E$ decoded. \\
        
        \(
        \begin{array}{ll}
        	M & \equiv E^{d} $ $ ($mod $ 221) \\
              & \equiv 2^{17} $ $ ($mod $ 221) \\
              & \equiv 19 $ $ ($mod $ 221) \\
        \end{array} \\ \)
        
        \item
        $p - 1 = 12$ \\
        $q - 1 = 16$
        
        \( \\
        \begin{array}{llll}
        	113 & = 9 \cdot 12 & + & 5 \\
            12 & = 2 \cdot 5 & + & 2 \\
            5 & = 2 \cdot 2 & + & 1 \\
        \end{array} \\ \)
        
        \(
        \begin{array}{ll}
        	5 = 2 \cdot 2 + 1 & \Rightarrow 5 - 2 \cdot 2 = 1 \\
            				  & \Rightarrow 113 - 9 \cdot 12 - 2(12 - 2 \cdot 5) = 1 \\
                              & \Rightarrow 113 - 11 \cdot 12 + 4(113 - 9 \cdot 12) = 1 \\
							  & \Rightarrow 5 \cdot 113 - 47 \cdot 12 = 1 \\
        \end{array} \\ \)
        
        $e^{-1} \equiv 113^{-1} \equiv 5$ (mod $12$) \\
        Let $d_{1} = 5$. \\
        
        $113 = 7 \cdot 16 + 1 \Rightarrow 113 - 7 \cdot 16 = 1$
        
        $e^{-1} \equiv 113^{-1} \equiv 1$ (mod $16$) \\
        Let $d_{2} = 1$.
        
        \( \\
        \begin{array}{ll}
        	\left\{
    			\begin{array}{ll}
        			x \equiv E^{d_{1}} $ $ ($mod $ p) \\
        			x \equiv E^{d_{2}} $ $ ($mod $ q)
    			\end{array}
			\right.
        	& \Rightarrow
            \left\{
    			\begin{array}{ll}
        			x \equiv 2^{5} $ $ ($mod $ 13) \\
        			x \equiv 2^{1} $ $ ($mod $ 17)
    			\end{array}
			\right. \\
            
            & \Rightarrow
            \left\{
    			\begin{array}{ll}
        			x \equiv 6 $ $ ($mod $ 13) \\
        			x \equiv 2 $ $ ($mod $ 17)
    			\end{array}
			\right.
        \end{array} \\ \)
        
        \(
        \begin{array}{ll}
        	17 & = 13 + 4 \\
			13 & = 3 \cdot 4 + 1 \\
        \end{array} \\ \)
        
        \(
        \begin{array}{ll}
        	13 = 3 \cdot 4 + 1 & \Rightarrow 13 - 3 \cdot 4 = 1 \\
            				   & \Rightarrow 13 - 3(17 - 13) = 1 \\
                               & \Rightarrow 4 \cdot 13 - 3 \cdot 17 = 1
        \end{array} \\ \)
        
        By the Chinese Remainder Theorem,
        
        \(
        \begin{array}{ll}
        	M & \equiv x $ $ ($mod $ 221) \\
              & \equiv 6(-3 \cdot 17) + 2(4 \cdot 13) $ $ ($mod $ 221) \\
              & \equiv -202 $ $ ($mod $ 221) \\
              & \equiv 19 $ $ ($mod $ 221) \\
        \end{array} \\ \)
        
    \end{myEnumerate}

\end{myEnumerate}

\end{document}
