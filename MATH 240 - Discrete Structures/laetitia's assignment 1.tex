\documentclass{article} 

\usepackage{amsmath,amsthm,amssymb,enumerate,dsfont,textpos}

\newtheorem{problem}{Problem} 

\theoremstyle{definition} 

\newtheorem*{solution}{Solution} 

\begin{document} \title{Assignment 1} 

\author{Laetitia Fesselier, ID 260791354} 

\date{\today}

\maketitle

\begin{problem} 
Use a truth table to determine if each statement is a tautology, contradiction, or contingency.
\end{problem}

\begin{enumerate}[(a)]
\item \((P \lor Q) \Rightarrow \neg P\)
\end{enumerate}


\begin{solution}
\end{solution}

\begin{tabular}{|c|c|c|c|c|}
  \hline
  P & Q & P $\lor$ Q & $\neg$ P &   P $\lor$ Q  $\Rightarrow$ $\neg$ P \\
  \hline
  T & T & T & F & F \\
  T & F & T & F & F \\
  F & T & T & T & T \\
  F & F & F & T & T \\
  \hline
\end{tabular}
\medskip \par

The statement is a contingency.\\

\begin{enumerate}[(b)]
\item \(( P \Leftrightarrow Q) \land (Q \Leftrightarrow R) \Rightarrow (P \Leftrightarrow R)\)
\end{enumerate}

\begin{solution}
\end{solution}

\begin{tabular}{|c|c|c|c|c|c|c|c|}
  \hline
  P & Q & R & P $\Leftrightarrow$ Q & Q $\Leftrightarrow$ R &  P $\Leftrightarrow$ R & (P $\Leftrightarrow$ Q) $\land$ (Q $\Leftrightarrow$ R) & (b) \\
  \hline
  T & T & T & T & T & T & T & T \\
  T & T & F & T & F & F & F & T \\
  T & F & T & F & F & T & F & T \\
  T & F & F & F & T & F & F & T \\
  F & T & T & F & T & F & F & T \\
  F & T & F & F & F & T & F & T \\
  F & F & T & T & F & F & F & T \\
  F & F & F & T & T & T & T & T \\
  \hline
\end{tabular}
\medskip \par

The statement is a tautology.\\

\pagebreak
\begin{enumerate}[(c)]
\item \( [(P \oplus Q) \oplus \neg Q] \Leftrightarrow P\)
\end{enumerate}

\begin{solution}
\end{solution}

\begin{tabular}{|c|c|c|c|c|c|}
  \hline
  P & Q & P $\oplus$ Q & $\neg$ Q & (P $\oplus$ Q) $\oplus$ $\neg$Q & [(P $\oplus$ Q) $\oplus$ $\neg$Q] $\Leftrightarrow$ P  \\
  \hline
  T & T & F & F & F & F \\
  T & F & T & T & F & F \\
  F & T & T & F & T & F \\
  F & F & F & T & T & F \\
  \hline
\end{tabular} 
\medskip \par
The statement is a contradiction.\\

\begin{problem}
Verify the following statements using only identities. Show
all of your work and name the identity or identities used in each step.
\end{problem}

\begin{enumerate}[(a)]
\item \( [(P \Rightarrow Q) \land P] \Rightarrow Q\) is a tautology 
\end{enumerate}

\begin{solution}
\end{solution}

\(
\begin{array}{llr}
	[P \Rightarrow Q) \land P] \Rightarrow Q & \equiv (\neg P \lor Q) \land P \Rightarrow Q & \text{(conditional)} \\
	&\equiv (\neg P \land P) \lor (Q \land P) \Rightarrow Q & \text{(distributivity)} \\
	&\equiv \mathds{F} \lor (Q \land P) \Rightarrow Q  & \text{(complement)} \\
	&\equiv Q \land P \Rightarrow Q & \text{(identity)} \\
	&\equiv \neg (Q \land P) \lor Q & \text{(conditional)} \\
	&\equiv \neg Q \lor \neg P \lor Q & \text{(DeMorgan's law)} \\
	&\equiv \neg Q \lor Q \lor \neg P & \text{(commutativity)} \\
	&\equiv \mathds{T} \lor \neg P & \text{(complement)} \\
	&\equiv \mathds{T} & \text{(domination)} \\
	&&\qed
\end{array}
\)

\begin{enumerate}[(b)]
\item \( \neg (P \land Q) \land (Q \Rightarrow P) \equiv \neg Q \)
\end{enumerate}

\begin{solution}
\end{solution}

\(
\begin{array}{llr}
	\neg (P \land Q) \land (Q \Rightarrow P) &\equiv \neg (P \land Q) \land (\neg Q \lor P) & \text{(conditional)} \\
	&\equiv (\neg P \lor \neg Q) \land (\neg Q \lor P) & \text{(DeMorgan's law)} \\
	&\equiv \neg Q \lor (\neg P \land P) & \text{(commutativity, distributivity)} \\
	&\equiv \neg Q \lor \mathds{F} & \text{(complement)} \\
	&\equiv \neg Q & \text{(identity)} \\
	&&\qed
\end{array}
\)

\pagebreak

\begin{enumerate}[(c)]
\item \( \neg[(P \lor Q) \lor [(Q \lor \neg R) \land (P \lor R)]] \equiv \neg P \land \neg Q \)
\end{enumerate}

\begin{solution}
\end{solution}

\(
\begin{array}{llr}
	& \neg[(P \lor Q) \lor [(Q \lor \neg R) \land (P \lor R)]] \\
	\equiv & \neg[(P \lor Q) \lor [[(Q \lor \neg R) \land P] \lor [(Q \lor \neg R) \land R)]]] & \text{(distributivity)} \\
	\equiv & \neg[(P \lor Q) \lor [[(Q \land P) \lor (\neg R \land P)] \lor [(Q \land R) \lor (\neg R \land R)]]] & \text{(distributivity)} \\
	\equiv & \neg[(P \lor Q) \lor [[(Q \land P) \lor (\neg R \land P)] \lor [(Q \land R) \lor \mathds{F}]]] & \text{(complement)} \\
	\equiv & \neg[P \lor Q \lor (Q \land P) \lor (\neg R \land P) \lor (Q \land R)] & \text{(associativity, identity)} \\
	\equiv & \neg[P \lor Q \lor (\neg R \land P) \lor (Q \land R)] & \text{(absorption)} \\
	\equiv & \neg[P \lor (\neg R \land P) \lor Q \lor (Q \land R)] & \text{(commutativity)} \\ 
	\equiv & \neg(P \lor Q) & \text{(absorption)} \\ 
	\equiv & \neg P \land \neg Q & \text{(DeMorgan's law)} \\
	&&\qed
\end{array}
\)

\begin{problem}
Of the following conditional and biconditional statements, which are true and which are false?
Briefly justify your answers.
\end{problem}

\begin{enumerate}[(a)]
\item $\pi$ is an integer if and only if $\sqrt{e + 3}$ is a vowel.
\end{enumerate}

\begin{solution}
\end{solution}

P: $\pi$ is an integer,  P \(\equiv \mathds{F}\) \par
Q: $\sqrt{e + 3}$ is a vowel, Q \(\equiv \mathds{F}\)

Symbolization of (a) : \(P \Leftrightarrow Q \equiv \mathds{F} \Leftrightarrow \mathds{F} \equiv \mathds{T} \) \par
The statement is true.

\begin{enumerate}[(b)]
\item 0 $>$ 1 whenever 2 + 2 = 4.
\end{enumerate}

P: 0 $>$ 1,  P \(\equiv \mathds{F}\) \par
Q: 2 + 2 = 4, Q \(\equiv \mathds{T}\)

Symbolization of (b): \(Q \Rightarrow P \equiv \mathds{T} \Rightarrow \mathds{F} \equiv \mathds{F} \) \par
The statement is false.

\begin{enumerate}[(c)]
\item If (a) implies (b), then pigs cannot fly.
\end{enumerate}

P: (a) implies (b),  P \(\equiv (a) \Rightarrow (b) \equiv  \mathds{T} \Rightarrow \mathds{F} \equiv  \mathds{F} \) \par
Q: Pigs cannot fly, Q \(\equiv \mathds{T}\)

Symbolization of (c): \([(a) \Rightarrow (b)] \Rightarrow Q \equiv \mathds{F} \Rightarrow \mathds{T} \equiv \mathds{T} \) \par
The statement is true.

\begin{problem}
Symbolize the following English sentences in logic, using the abbreviation scheme provided.
\end{problem}

\begin{enumerate}[(a)]
\item \text{"Thunder only happens when it's raining."}
\end{enumerate}
\text{T : thunder happens} \\
\text{R : it's raining} \\
Symbolization of (a): \( T \Rightarrow R \)

\begin{enumerate}[(b)]
\item "For every positive integer n there is a prime number that is bigger than n but at most 2n."
\end{enumerate}
\text{I(x) : x is a positive integer} \\
\text{P(x) : x is a prime number} \\
\text{B(x, y) : x is bigger than y} \\
Symbolization of (b): \( \forall n [I(n) \Rightarrow \exists x (P(x) \land B(x,n) \land B(2n+1, x ))]  \)

\begin{enumerate}[(c)]
\item "Goldbach's Conjecture is true if every even integer greater than 2 can be written as the sum of two primes."
\end{enumerate}
\text{G : Goldbach's Conjecture is true} \\
\text{E(x) : x is an even integer} \\
\text{T(x) : x is greater than 2} \\
\text{P(x) : x is the sum of two primes} \\
Symbolization of (c): \( \forall x ( E(x) \land T(x) \land P(x)) \Rightarrow G \)




\end{document}